% !TEX root = ../../main.tex
\section{Results}\label{sec:real_world_results}

In the first paragraph we will explain that we use the same quality measures as used in \Cref{sec:artificial_data_quality_metrics} for objective tabular results.
We further argue that more important is the subjective, visual, inspection of the discovered change points regarding the sensor data.
To give both results, we will first give a table and box plot, like in \Cref{sec:artificial_data_results}.
After that, we will show characterizing parts of the plots with annotated and discovered change points.

\subsection{Objective measure}

\TODO{Table of search run (in the columns) and for each run the ratio of False Alarm Rate, Average benefit (closest CP) and STD benefit. Like chapter 5.}

\begin{table}
  \centering
  \begin{tabulary}{\textwidth}{|l|c|c|c|c|c|c|c|c|}
    \cline{2-9}
    \multicolumn{1}{l|}{} & Run 1 & Run 2 & Run 3 & Run 4 & Run 5 & Run 6 & Run 7 & Run 8 \\
    \hline
    Window length & 0 & 0 & 0 & 0 & 0 & 0 & 0 & 0 \\
    \hline
    Sigma of \gls{rbf} & 0 & 0 & 0 & 0 & 0 & 0 & 0 & 0 \\
    \hline
    High threshold & 0 & 0 & 0 & 0 & 0 & 0 & 0 & 0 \\
    \hline
    Low threshold & 0 & 0 & 0 & 0 & 0 & 0 & 0 & 0 \\
    \hline
    Closeness & 0 & 0 & 0 & 0 & 0 & 0 & 0 & 0 \\
    \hline
    \hline
    $\operatorname*{far}(Y)$ & 0 & 0 & 0 & 0 & 0 & 0 & 0 & 0 \\
    \hline
    $\operatorname*{Average\_delay}$ & 0 & 0 & 0 & 0 & 0 & 0 & 0 & 0 \\
    \hline
    STD Delay & 0 & 0 & 0 & 0 & 0 & 0 & 0 & 0 \\
    \hline
  \end{tabulary}
  \caption[Results real world runs]{Parameter settings and results of the real-world data sets.}
  \label{tab:results_real_world}
\end{table}

\TODO{Box plot of the runs, like in Chapter 5}

Here comes a list of observations and conclusions on the performance:
\begin{itemize}
  \item \textbf{Perhaps something in bold:} here we list a funny observation.
  \eg that when walking the circulair stairs it is harder to find the rotation on the flat surface.
  \item Or just something else.
\end{itemize}

\subsection{Subjective measures}
In this subsection we will provide a few plots with characterizing parts of the sensor data, annotated change points, and discovered change points.
We use it to illustrate some aspects of the method, on which it performs well and where not.
We do not provide all plots, because that would take up to much time.
We do try to give a subjective conclusion about the performance, backed by the provided examples.