% !TEX root = ../../main.tex
\section{Change-detection by Density-Ratio Estimation}\label{density-ratio}

Papers:
\begin{itemize}
  \item Change-Point Detection in Time-Series Data by Direct Density-Ratio Estimation, 2009, 45 refs \cite{kawahara2009change}
  \item Change-Point Detection in Time-Series Data by Relative Density-Ratio Estimation, 2013, 3 refs \cite{liu2013change}
  \item Density ratio estimation in machine learning (Book), 2009, 24 refs \cite{sugiyama2012density}
\end{itemize}

Formulate the problem of detecting change in the statistical framework.
Consider the probability distributions from which two consecutive segments of time series around a target time point are generated.
When the disitrubtions differ significantly the target time point is regarded as a change point.




CUSUM (cumulative sum) \cite{basseville1993detection} and GLR (generalized likelihood ratio)


The distribution over the values of time series data can be represented with a probability density function (pdf).
Two sections of a time series data can be generated with the same underlying pdf or each with a different.
