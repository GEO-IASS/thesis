% !TEX root = ../main.tex
% Chapter 2

\chapter{Literature review}

\label{Chapter2} % For referencing the chapter elsewhere, use~\ref{Chapter2}

\lhead{Chapter 2. \emph{Literature review}} % This is for the header on each page - perhaps a shortened title

%----------------------------------------------------------------------------------------
% \section{Outline}
% \emph{Not intented for the reader.}
% \begin{itemize}
%   \item Literature review about Temporal Segmentation (previous draft was more about classification)
%   \item Consider methods for the context of filter-methods for classification
%   \item Take a loot at 3-4 different kind of methods for change detection:
%     \begin{itemize}
%       \item Introduction with a lot of techniques
%       \item Explain why look at a few
%       \item CUSUM - or other more traditional methods
%       \item Density-ratio estimation
%       \item Support Vector Machines (?) - if there are more sources about this
%       \item (Dimensionality reduction) --> probably not
%       \item \emph{Not to much attention to all techniques, focus is on SVM}
%     \end{itemize}
%   \item With each method, shortly look at characteristics, strengths and weaknesses and consider applicability to accelerometer sensor data
% \end{itemize}


% \section{Statistical framework}\label{statistical-framework}


% -- Nieuwe indeling Chapter 2 --

% \begin{description}
%   \item[Intro/outline] Introduction and overview of this chapter. Put in the context of temporal classification.
%   \item[Temporal segmentation] Give overview of used temporal segmentation methods, move towards a split in model building/dimensionality reduction and change detection methods
%   \item[Dimensionality reduction] Explain that \gls{dr} can be useful in data processing. Name a few techniques, make bridge to \gls{oc-svm}.
%   \item[Change detection] Describe methods for \emph{direct} change detection. End with notion
%     \begin{description}
%       \item[Model fitting]
%       \item[Cusum]
%       \item[Density ratio]
%     \end{description}
%   \item[Support Vector Machine based methods] test

% \end{description}
As described in \Cref{Chapter1}, the focus of this research is the application of temporal segmentation to accelerometer data.
This is with the assumption that explicit temporal segmentation can aid the process of temporal classification.

In recent years a lot of research has been performed to study classification of human activities, recorded by on-body sensors.
Especially since smartphones with accelerometer sensors became widely available, research has focused on recordings from these devices.

In \Cref{sec:literature_review_statistical_models} we will start with an overview of statistical models used in this field of study.
That section will act as a starting point to look at the different types, applications and methods of temporal segmentation.
\Cref{sec:literature_review_temporal_segmentation} follows the distinction made and looks at a collection of temporal segmentation and change detection methods.
It will also relate the concept of \gls{occ} to temporal segmentation and change detection.
In \Cref{sec:density-ratio} methods using density-ratio estimation of \glspl{pdf} for change detection are discussed.
The final section of this chapter, \Cref{sec:literature_review_svm}, will discuss a specific class of algorithms for \gls{occ}, namely the methods that use \glspl{svm} for the classification task.

% !TEX root = ../../main.tex
\section{Statistical models}\label{sec:literature_review_statistical_models}

\TODO{Introduce the diffent models and thus different approaches}

Many applications require the detection of time points at which the underlying properties of a system change.
This problem has received a lot of attention in the fields of data mining, etc... \TODO{list and refs}.
Often this problem is formulated in a statistical framework, by inspecting the underlying data generating \gls{pdf} of the time series data.
A change point is then defined as a significant change in the properties of the \gls{pdf}, such as the mean and variance.

The widely used \gls{cusum} method by Basseville \etal \cite{basseville1993detection} follows this approach.
It originates from control methods for detection from benchmarks.
This method and some derivatiges are discussed and analyzed in \Cref{sec:cusum}.

Many methods rely on pre-specified parametric model assumptions and considers the data be independent over time, which makes it less flexible to real-world applications.
The methods proposed by Kawahara \etal \cite{kawahara2009change} and Lui \etal \cite{liu2013change} try to overcome these problems by estimating the \emph{ratio} between the \glspl{pdf}, instead of estimating each \gls{pdf}.
This approach is discussed and analyzed in \Cref{sec:density-ratio}.

The density-estimation methods rely on the log-likelihood ratio between \glspl{pdf}.
The method of Camci \cite{camci2010change} follows an other approach within the statistical framework, by using a \gls{svm}.
One problem it tries to overcome is the (claimed) weakness of many methods to detect a decrease in variance.
The method represents the distribution over the data points as a hyper-sphere in a higher dimension using the kernel trick.
A change in the \gls{pdf} is represented by a change in the radius of this sphere.
\Cref{sec:literature_review_svm} discusses the \gls{svm}-methods.
\TODO{Put emphasis that this is the source of inspiration for the chosen method?}


-- NEW --

In the field of temporal segmentation of time series data we can distinguish different approaches and common frameworks in which they perform.
The problem is commonly formulated in the statistical framework and regarding the properties of the underlying data generating \gls{pdf} fo the time series data.
In that sense, a change point is then defined as a significant change in the properties of the \gls{pdf}, such as the mean or variance.

In the search for the change in properties, temporal segmentation and change point detection methods can roughly be categorized in three methods in the way the data is processed, as discussed by Avci \etal \cite{avci2010activity}:
\begin{itemize}
  \item \textbf{Top-Down} methods iteratively divide the signal in segments by splitting at the best location.
  The algorithm starts with two segments and completes when a certain condition is met, such as when an error value or number of segments $k$ is reached.
  % These methods process the data points recursively, which results in a complexity of $O(n^2k)$.
  \item \textbf{Bottom-Up} methods are the natural complement to top-down methods.
  They start with creating $n/2$ segments and iteratively join adjacent segments while the value of a cost function for that operation is below a certain value.
  \item \textbf{Sliding-window} methods are simple and intuitive for online segmenting purposes.
  It starts with a small initial subsequence of the time series.
  New data points are joined in the segment until the fit-error is above a threshold.
  Since the data is only processed very locally, these methods can yield in poor results \cite{keogh2001online}.
  \item \textbf{\acrlong{swab}}, as introduced by Keogh \etal \cite{keogh2001online}, combines the ability of the sliding window mechanism to process time series online and the bottom-up approach the create superior segments in terms of fit-error.
  The algorithm processes the data in two stages.
  The first stage is to join new data points in the current segment created by a sliding window.
  The second stage processes the data using Bottom-Up and return the first segment as the final result.
  Since this second stage retains some (semi-)global view of the data, the results are comparative with normal Bottom-Up.
\end{itemize}
It is clear that for the application of this research sliding-window and preferably \gls{swab}-based algorithms should be considered.
In the following we will dicuss classes of algorithms grouped by the type of decision function, assuming a \gls{swab}-based data processing order.

% CUSUM
A widely used class of methods is based on the \gls{cusum} approach by Basseville \etal \cite{basseville1993detection}.
This approach originates from control methods for detection and faults and benchmarking.
In \Cref{subsec:literature_review_temporal_segmentation_cusum} a further overview of this class of methods is discussed.


% Piecewise approximation


-- Density

-- Outlier based
% !TEX root = ../../main.tex
\section{Temporal Segmentation}\label{sec:temporal_segmentation}

\begin{itemize}
  \item Given overview of segmentation techniques, for times series data
  \item Use different ``point-of-views'', or terms
  \item ``Segmentation''
  \item ``Change detection''
  \item ``Novelty detection''
  \item Specific view on \glspl{svm}
\end{itemize}

This section gives an overview of the literature on temporal segmentation in the context of \gls{har}.
It takes a look on different implementations and methodologies.
A wide range of terms and subtle differences are used in the field, such as `segmentation', `change detection', `novelty detection' and `outlier detection'.
These will be the categorical terms for which we discuss the literature.
Finally we will discuss other applications of \glspl{svm} in the context of these terms.

\subsection{Segmentation}\label{subsec:segmentation}
\TODO{This subsection is mainly from previous draft version}

\TODO{Create compact notation, one sentence per paper max}

Segmentation methods can roughly be categorized in three methods in the way the data is processed, as discussed by Avci \etal \cite{avci2010activity}:
\begin{itemize}
  \item \textbf{Top-down} methods iteratively divide the signal in segments by splitting at the best location.
  The algorithm starts with two segments and completes when a certain condition is met, such as when an error value or number of segments $k$ is reached.
  These methods process the data points recursively, which results in a complexity of $O(n^2k)$.
  \item \textbf{Bottom-up} methods are the natural complement to top-down methods.
  They start with creating $n/2$ segments and iteratively join adjacent segments while the value of a cost function for that operation is below a certain value.
  Given the average segment length $L$, the complexity of this method is $O(nL)$.
  \item \textbf{Sliding-window} methods are simple and intuitive for online segmenting purposes.
  It starts with a small initial subsequence of the time series.
  New data points are joined in the segment until the fit-error is above a threshold.
  Since the data is only processed very locally, these methods can yield in poor results \cite{keogh2001online}.
  The complexity is equal to the bottom-up approach, $O(nL)$, where $L$ is the average segment length.
  \item \textbf{\acrlong{swab}}, as introduced by Keogh \etal \cite{keogh2001online}, joins the ability of the sliding window mechanism to process time series online and the bottom-up approach the create superior segments in terms of fit-error.
  The algorithm processes the data in two stages.
  The first stage is to join new data points in the current segment created by a sliding window, and pass this to a buffer with space for a few segments.
  The buffer then processes the data Bottom-up and returns the first (left-most) segment as final segment.
  Because this second stage retains some (semi-)global view of the data, the results are comparative with normal Bottom-up.
  It is stated by Keogh \etal that the complexity of \gls{swab} is a small constant factor worse than that of regular Bottom-up.
\end{itemize}
It is clear that for the application of this research sliding-window and preferably \gls{swab}-based algorithms should be considered.

The \gls{swab} method proposed by Keogh \etal \cite{keogh2001online} is dependent on an user setting, providing the maximum error when performing both stages.
Each segment is approximated by using piecewise linear representation (PLR), an often used method.
The user provided error threshold controls the granularity and number of segments.
Other methods have been proposed, such as an adaptive threshold based on the signal energy by Guenterberg \etal \cite{guenterberg2009automatic}, the adaptive \gls{cusum}-based test by Alippi \etal \cite{alippi2006adaptive} and the \gls{mosum} by Hsu \cite{hsu2007mosum} in order to eliminate this user-dependency.
The latter of these methods is able to process the accelerometer values directly, although better results are obtained when features of the signal are processed, as done in the former method.
Here the signal energy, mean and standard deviation are used to segment activities and by adding all the axial time series together, the Signal-To-Noise ration is increased, resulting in a robuster method.

The method of Guenterberg \etal extracts features from the raw sensor signal to base the segmentation on other properties than the pure values.
The method of Bernecker \etal \cite{bernecker2012activity} uses other statistical properties, namely autocorrelation, to distinguish periodic from non-periodic segments.
Using the \gls{swab} method the self-similarity of a one-dimensional signal is obtained.
The authors claim that only a slight modification is needed to perform the method on multi-dimensional data.
After the segmentation phase, the method of Bernecker \etal extracts other statistical features which are used in the classification phase.

The proposal of Guo \etal \cite{guo2012adaptive} dynamically determines which features should be used for the segmentation and simultaneously determines the best model to fit the segment.
For each of the three dimensions features such as the mean, variance, covariance, correlation, energy and entropy are calculated.
By extending the \gls{swab} method, for every frame a feature set is selected, using an enhanced version of \gls{pca}.
The research also considered the (Stepwise) Feasable Space Window as introduced by \cite{liu2008novel}, but since it results in a higher error rate than \gls{swab} the latter was chosen to extend.
Whereas the before mentioned algorithms use a linear representation, this methods considers linear, quadratic and cubical representations for each segment.
This differs from other methods where the model is fixed for the whole time series, such as \cite{fuchs2010online}, which is stated to perform inferior on non-stationary time series such as daily life.

The time series data from a sensor can be considered as being drawn from a certain stochastic process.
Probabilistic models can be constructed on that signal, yielding in probabilistic and Bayesian based segmentation methods.
The \gls{cusum}-methods takes a statistical approach and relies on the log-likelihood ratio \cite{gustafsson1996marginalized} to measure the difference between two distributions.
To calculate the ratio, the probability density functions need to be calculated.
The method of Kawahara \etal \cite{kawahara2009change} proposes to estimate the ratio of probability densities (known as the \emph{importance}), based on the log likelihood of test samples, directly, without explicit estimation of the densities.
The method by Liu \etal \cite{liu2013change} uses a comparable dissimilarity measure using the \gls{kliep} algorithm.
They claim this results in a robuster approach for real-world scenarios.
Although this is a model-based method, no strong assumptions (parameter settings) are made on the models.

The method of Adams and MacKay \cite{adams2007bayesian} builds a probabilistic model on the segment run length, given the observed data so far.
Instead of modeling the values of the data points as a probabilistic distribution, the length of segments as a function if time is modeled by calculating the posterior probability.
It uses a prior estimate for the run length and a predictive distribution for newly-observed data, given the data since the last change point.
This method contrasts with the approach of Guralnik and Srivastava \cite{guralnik1999event} in which change points are detected by a change in the (parameters of an) underlying, observed, model.
For each new data point, the likelihoods of being a change point and part of the current segment are calculated, without a prior model (and thus is a non-Bayesian approach).
It is observed that when no change point is detected for a long period of time, the computational complexity increases significantly.

Another application of \gls{pca} is to characterize the data by determining the dimensionality of a sequence of data points.
The proposed method of Berbi\v{c} \etal \cite{barbivc2004segmenting} determines the number of dimensions (features) needed to approximate a sequence within a specified error.
With the observation that more dimensions are needed to keep the error below the threshold when transitions between actions occur, cut-points can be located and segments will be created.
The superior extension of their approach uses a Probabilistic \gls{pca} algorithm to incorporate the dimensions outside the selected set as noise.

In the method by Himberg \etal \cite{himberg2001time} a cost function is defined over segments of data which is to be minimized.
The cost functions thereby searches for internally homogeneous segments of data, reflecting states in which the devices and the user are.
The cost function can be any arbitrary function and in the implementation the sum of variances over the segments is used.
Both in a local and global iterative replacement procedure (as an alternative for the computationally hard dynamic programming algorithm) the best breakpoint locations $c_i$ for a pre-defined number of segments $1 \leq i \leq k$ are optimized.

Many methods obtain an implicit segmentation as a result of classification over a sliding window \TODO{add refs}.
The method of Yang \etal \cite{yang2008distributed} explicitly performs segmentation and classification simultaneously.
It argues that the classification of a pre-segmented test-sequences becomes straightforward with many classical algorithms to choose from.
The algorithm matches test examples with the \emph{sparsest} linear representation of mixture subspace models of training examples, searching over different temporal resolutions.

The method of Chamroukhi \etal \cite{chamroukhi2013joint} is based on a \gls{hmm} and logistic regression.
It assumes a $K$-state hidden process with a (hidden) state sequence, each state providing the parameters (amongst which the order) for a polynomial.
The order of the model segment is determined by model selecting, often using the \gls{bic} or the similar \gls{aic} \cite{akaike1974new}, as in \cite{he2008activity}.

Field of computer vision: \cite{zhou2008aligned}, \cite{li2007segmentation}.

--- Segmentation ---

% ``An online algorithm for segmenting time series'' \cite{keogh2001online}. See \ref{sec:appendix-C-online-keogh}. 538, 2001 \\
% ``Segmenting time series: A survey and novel approach'' \cite{keogh2004segmenting}. 242, 2004 \\

% ``Time series segmentation for context recognition in mobile devices'' \cite{himberg2001time}. 158, 2001 \\

% ``An Adaptive Approach for Online Segmentation of Multi-Dimensional Mobile Data'' \cite{guo2012adaptive}. 4, 2012 \\

% ``Joint segmentation of multivariate time series with hidden process regression for human activity recognition'' \cite{chamroukhi2013joint}. 0, 2013. See \ref{sec:appendix-C-joint-segmentation}. \\

``Segmentation and Recognition of Motion Streams by Similarity Search'' \cite{li2007segmentation}. 29, 2007 \\

% ``Novel Online Methods for Time Series Segmentation'' \cite{liu2008novel}. 22, 2008 \\

% ``Distributed Segmentation and Classification of Human Actions Using a Wearable Motion Sensor Network'' \cite{yang2008distributed}. 44, 2008 \\

% ``An Automatic Segmentation Technique in Body Sensor Networks based on Signal Energy'' \cite{guenterberg2009automatic}. 14, 2009 \\

% ``Online Segmentation of Time Series Based on Polynomial Least-Squares Approximations'' \cite{fuchs2010online}. 24, 2010 \\

``Aligned Cluster Analysis for Temporal Segmentation of Human Motion'' \cite{zhou2008aligned}. 63, 2008 \\

% ``Segmenting motion capture data into distinct behaviors'' \cite{barbivc2004segmenting}. 270, 2004 \\

\subsection{Change detection}\label{subsec:change_detection}
\TODO{change order of this and temporal segmentation sections? So first change, then segmentation?} \\
Whereas the above mentioned researches focus on \emph{segmentation}, many have focused on \emph{change detection}.
Although these techniques are closely related, there is a subtle difference.
In the case of \emph{change detection} to goal is to find, possibly unrelated, sudden change points in a signal \cite{takeuchi2006unifying}.
In contrast, the goal of \emph{temporal segmentation} is to find homogeneous segments of data, which can be the result of multiple detected changes.

The \gls{icss} by Incl\'{a}n and Tiao \cite{inclan1994use} is a statistical method which obtains results (when applied to stock data) comparable to \gls{mle} and Bayesian \TODO{Bayesian What?}.
Whereas \gls{cusum} can be applied to search for a change in mean, the \gls{icss} is adapted to find changes in variance.
It obtains a \emph{likelihood ratio} for testing the hypothesis of one change against no change in the variance.
Using an iterative approach, all possible change points are considered.
The proposal of \cite{cheng2009efficient} extends on the \gls{cusum}-based methods to find change points in mean and variance, by creating a more efficient and accurate algorithm.

Section \ref{sec:change_detection_time_series} discusses the problem of change detection in time series further, and gives a formal problem definition.

\TODO{move CUSUM based techniques to this subsection}


--- Change detection ---

``A unifying framework for detecting outliers and change points from time series'' \cite{takeuchi2006unifying}. 87, 2006 \\

% ``Bayesian online changepoint detection'' \cite{adams2007bayesian}. 85, 2007 \\

% ``An adaptive cusum-based test for signal change detection'' \cite{alippi2006adaptive}. 18, 2006 \\

% ``An efficient algorithm for estimating a change-point'' \cite{cheng2009efficient}. 5, 2009 \\

% ``The marginalized likelihood ratio test for detecting abrupt changes'' \cite{gustafsson1996marginalized}. 80, 1996 \\

% ``The MOSUM of squares test for monitoring variance changes'' \cite{hsu2007mosum}. 4, 2007 \\

% ``Use of cumulative sums of squares for retrospective detection of changes of variance'' \cite{inclan1994use}. 643, 1994 \\

% ``Change-point detection in time-series data by direct density-ratio estimation'' \cite{kawahara2009change}. 52, 2009 \\
``Sequential change-point detection based on direct density-ratio estimation'' \cite{kawahara2012sequential}. 22, 2012 \\
% ``Change-point detection in time-series data by relative density-ratio estimation'' \cite{liu2013change}. 11, 2013 \\

``Change point detection in time series data using support vectors'' \cite{camci2010change}. 3, 2010 \\

\subsection{Novelty detection}\label{subsec:novelty_detection}

--- Novelty detection ---

``Online novelty detection on temporal sequences'' \cite{ma2003online}. 146, 2003 \\
``Time-series novelty detection using one-class support vector machines'' \cite{ma2003time}. 78, 2003 \\

``Novelty detection: a review—part 1: statistical approaches'' \cite{markou2003novelty}. 697, 2003 \\

``Support Vector Method for Novelty Detection'' \cite{scholkopf1999support}. 337, 1999 \\

\subsection{Outlier detection}\label{subsec:outlier_detection}

--- Outlier detection ---
``A unifying framework for detecting outliers and change points from time series'' \cite{takeuchi2006unifying}. 87, 2006 \\

``Outliers in statistical data'' \cite{barnett1994outliers} (book). 3745, 1994 \\

``A survey of outlier detection methodologies'' \cite{hodge2004survey}. 791, 2004 \\

``Local outlier detection reconsidered: a generalized view on locality with applications to spatial, video, and network outlier detection'' \cite{schubert2012local}. 2, 2012 \\


% --- SVMs ---

% ``Change point detection in time series data using support vectors'' \cite{camci2010change}. 3, 2010 \\

% ``Time-series novelty detection using one-class support vector machines'' \cite{ma2003time}. 78, 2003 \\

% ``Support vector domain description'' \cite{tax1999support}. 907, 1999
% ``Support vector data description applied to machine vibration analysis'' \cite{tax1999supportdata}. 76, 1999 \\

% ``Human Activity Recognition on Smartphones using a Multiclass Hardware-Friendly Support Vector Machine'' \cite{anguitahuman}. 13, 2012 \\

% ``Support vector machines in remote sensing: A review'' \cite{mountrakis2011support}. 150, 2011 \\

% ``Sensor-based abnormal human-activity detection'' \cite{yin2008sensor}. 74, 2008 (builds one-class SVM of all normal traces) \\
% % !TEX root = ../../main.tex
\section{Change Detection methods}\label{sec:change_detection_methods}
Describe methods for \emph{direct} change detection. End with notion that all work on low/single dimensional time series data, so dimensionality reduction is needed.

\subsection{CUSUM}

\subsection{Model fitting}

\subsection{Density-Ratio}

% -- OLD --

%----------------------------------------------------------------------------------------
% CUSUM and GLR
% % !TEX root = ../../main.tex
\section{CUSUM}\label{sec:cusum}

% \TODO{read book Basseville \cite{basseville1993detection}}


% Papers:
% \begin{itemize}
%   \item Use of Cumulative Sums of Squares for Retrospective Detection of Changes of Variance, 1994, 162 refs \cite{inclan1994use} \\
%   Implemented this algorithm.
%   \item An adaptive CUSUM-based test for signal change detection, 2006, 15 refs. \cite{alippi2006adaptive}
%   \item The MOSUM of squares test for monitoring variance changes \cite{hsu2007mosum}
% \end{itemize}

\emph{Notes:}
Non-Bayesian change detection algorithm (\ie no prior distribution believe available for the change time).
The \gls{cusum} method is developed by Page \cite{page1954continuous} for the application of statistical quality control (it is also known as a control chart).
Primary for detection of mean shift.
The \gls{mosum} of squares test for monitoring variance changes \cite{hsu2007mosum}.
Use of Cumulative Sums of Squares for Retrospective Detection of Changes of Variance \cite{inclan1994use}

An often used approach in the statistical framework of change detection is the \gls{cusum} as introduced by Page \cite{page1954continuous}.
Originally used for quality control in production environments, its main function is to detect change in the mean of measurements and has been applied to this problem \cite{basseville1993detection}.
It is a non-Bayesian method and thus makes no assumptions, in the form of prior belief distributions, for the change points.
Many extensions to this method have been proposed.
Some focus on the change in mean, such the method of Alippi and Roveri \cite{alippi2006adaptive}.
Others apply the method the problems in which the change of variance is under consideration.
Among others are there the centered version of the cumulative sums, introduced by Brown, Durbin and Evans \cite{brown1975techniques} and the \gls{mosum} of squares by \cite{hsu2007mosum}.

The method of Incl\'{a}n and Tiao \cite{inclan1994use} builds on the centered version of \gls{cusum} \cite{brown1975techniques} to detect changes in variance.
Using the batch \gls{icss} algorithm they are able to find multiple change points, offline while post-processing the data.
Let $C_k = \sum_{i=1}^k \alpha_t^2$ be the cumulative sum of squares for a series of uncorrelated random variables $\{\alpha_t\}$ of length $T$.
The centered (and normalized) sum of squares is defined as
\begin{equation}
  \begin{aligned}
  D_k = \frac{C_k}{C_T} - \frac{k}{T}, & & k = 1, \dots, T, & & \text{with } D_0 = D_T = 0.
  \end{aligned}
\end{equation}
For a series with homogeneous variance, the value of $D_k$ will oscillate around $0$.
In case of a sudden change, the value will increase and exceed some predefined boundary with high probability.
By using an iterative algorithm, the method is able to minimize the masking effect of successive change points.

One of the motivations for the \gls{icss} algorithm was the heavy computational burden involved with Bayesian methods, which need to calculate the posterior odds for the log-likelihood ratio testing.
The \gls{icss} algorithm avoids applying a function at all possible locations, due to the iterative search.
The authors recommend the algorithm for analysis of long sequences.

%----------------------------------------------------------------------------------------
% Density Ratio Estimation
% !TEX root = ../../main.tex
\section{Change-detection by Density-Ratio Estimation}\label{sec:density-ratio}

% Papers:
% \begin{itemize}
%   \item Change-Point Detection in Time-Series Data by Direct Density-Ratio Estimation, 2009, 45 refs \cite{kawahara2009change}
%   \item Change-Point Detection in Time-Series Data by Relative Density-Ratio Estimation, 2013, 3 refs \cite{liu2013change}
%   \item Density ratio estimation in machine learning (Book), 2009, 24 refs \cite{sugiyama2012density}
%   \item Direct importance estimation for covariate shift adaptation, 2008, 84 refs \cite{sugiyama2008direct}
% \end{itemize}

% Formulate the problem of detecting change in the statistical framework.
% Consider the probability distributions from which two consecutive segments of time series around a target time point are generated.
% When the disitrubtions differ significantly the target time point is regarded as a change point.




% CUSUM (cumulative sum) \cite{basseville1993detection} and GLR (generalized likelihood ratio)


% The distribution over the values of time series data can be represented with a probability density function (pdf).
% Two sections of a time series data can be generated with the same underlying pdf or each with a different.

Many approaches to detect change points monitor the logarithm of the likelihood ratio between two consecutive intervals.
A change point is regarded to be the moment in time when the underlying probabilistic generation function changes.
Some methods which rely on this are novelty detection, maximum-likelihood estimation and online learning of autoregressive models \cite{kawahara2009change}.
A limitation of these methods is that they rely on pre-specified parametric models.
Nonparametric models, for which the number and nature of the parameters are undetermined, for density estimation have been proposed, but it is said to be a hard problem \cite{hardle2004nonparametric, sugiyama2012density}.
A solution to this is to estimate the \emph{ratio} of probabilities instead of the probabilities themselves.
One of the recent methods to achieve this is the \gls{kliep} by Sugiyama \etal \cite{sugiyama2008direct}.

The proposed method by Kawahara and Sugiyama~\cite{kawahara2009change} uses an online version of the \gls{kliep} algorithm.
It considers \emph{sequences} of samples (rather than samples directly) because the time series samples are generally not independent over time.
The method detects change by monitoring the logarithm of the likelihood ratio between densities of reference (past) and test (current) time intervals.
If it exceeds a predetermined threshold value, the beginning of the test interval is marked as a change point.

Since the density ratio is unknown, it needs to be estimated.
The naive approach is to estimate it using estimated densities of the two intervals.
Since this is known to be a hard problem and sensitive for errors, the solution would be to estimate the ratio directly.

The method by Liu \etal \cite{liu2013change} estimates the ratio of probabilities directly, instead of estimating the densities explicitly.
Other methods using ratio-estimation are the Unconstrained Least-Squares Importance Fitting (uLSIF) method \cite{kanamori2009least}, and an extension which possesses a superior nonparametric convergence property: Relative uLSIF (RuLSIF) \cite{yamada2013relative}.

%----------------------------------------------------------------------------------------
% Support Vector Machines
% !TEX root = ../../main.tex
\section{Change-detection by Support Vector Machines}\label{svm}

Introduced by Vapnik \cite{vapnik1998statistical, vapnik1999nature}, Support Vector Machines offer a way to segment, and classify, linear separable data.
This technique can also be applied to estimate density functions of given time series \cite{weston1999support}.
When combined with a mapping function, which maps the data from the input space $I$ to a higher dimension feature space $F$, the input data can be non-linear separable.
The linear hyperplane, which segments the data in the feature space $F$, yields to a non-linear segmentation in the lower-dimensional input space $I$.
Instead of explicitly mapping the input data to the higher dimensional space, a kernel function can be used.
This kernel function can calculate values of the feature space directly, without the need to first map the input values to this space.
This process is referred to as the kernel trick.

The method of Camci \cite{camci2010change} uses one-class support vector machines to segment time series data.
For the current interval a hyper-sphere is constructed, using the kernel trick, which circumscribes all the data.
To cope with possible errors or outliers a soft-margin is applied.
The data can thus be represented using a center $c_1$ and radius $r_1$ of the hyper-sphere.
New data points are consecutively mapped from the input space to the feature space.
The retrieved (high dimension) data point can be in- or outside the earlier constructed hyper-sphere, thereby giving information about a possible change point.



\subsection{One-class Support Vector Machine}
*** This section is too detailed for the Literature Review, but keep it for now ***


*** Replace with / add information from summary in \ref{svm-explained}. ***

The proposed method of Camci \cite{camci2010change} uses a one-class support vector machine to segment time series data.
One-class SVMs are used to describe the current data under consideration, by assuming all data points are from the same class \cite{tax2001one}.
The class is described by a spherical boundary around the data with center $c$ and radius $r$, such that the volume is minimized.
Following the definition of Camci \cite{camci2010change}, the class description is obtained by minimizing $r^2$:
\begin{equation}
  \mathrm{Min}\ r^2
\end{equation}
\begin{equation}
  \mathrm{Subject\ to} : \|\mathbf{x}_i - \mathbf{c}\|^2 \le r^2\ \forall i,\ \mathbf{x}_i : i \mathrm{th\ data\ point}
\end{equation}

To be able to handle outliers in the input data, a penalty cost function $\varepsilon_i$ for each outlier can be added.

*** Add new function and constraints? ***

Using this one-class SVM formulation, differences between two (consecutive) windows of data points with size $w$ can be obtained.
The first window is used as the input set, $h_1$ and the second as the test set $h_t$.
For the first window a one-class SVM is constructed, yielding in a representation by $c_1$ and $r_1$.
When the data points of the second window belong to the same class, the representation of that one-class SVM would equal the first:
\begin{equation}
\begin{aligned}
  c_1 = c_2, & &  r_1 = r_2
\end{aligned}
\end{equation}

*** First tell more about (underlying) probability density functions, to relate to other methods ***

In case the second window of data points does not belong to the same class, i.e. the probability density function that describes the data differs from the first, the describing values of the second window will differ from the first.
The amount of difference can be expressed by a dissimilarity measure over the representations.
When the dissimilarity between the two windows exceeds some predefined threshold $th$, there exists a change point between the windows.

This process can be visualized as done in *** insert figure of four circles ***. The second window, $h_2$ can be constructed from the first by e.g. a shift of one data point. *** explain data point positions by circle ***.

Note that a difference in the SVM center $c$ or radius $r$ represent a change in the mean and variance, respectively.

