% !TEX root = ../main.tex
% Chapter 2

\chapter{Literature review}

\label{Chapter2} % For referencing the chapter elsewhere, use \ref{Chapter2}

\lhead{Chapter 2. \emph{Literature review}} % This is for the header on each page - perhaps a shortened title

%----------------------------------------------------------------------------------------
\section{Outline}
\begin{itemize}
  \item Literature review about Temporal Segmentation (previous draft was more about classification)
  \item Consider methods for the context of filter-methods for classification
  \item Take a loot at 3-4 different kind of methods for change detection:
    \begin{itemize}
      \item Introduction with a lot of techniques
      \item Explain why look at a few
      \item CUSUM - or other more traditional methods
      \item Density-ratio estimation
      \item Support Vector Machines (?) - if there are more sources about this
      \item (Dimensionality reduction)
    \end{itemize}
  \item With each method, shortly look at characteristics, strengths and weaknesses and consider applicability to accelerometer sensor data
\end{itemize}


\section{Statistical framework}\label{statistical-framework}
Many applications require the detection of time points at which the underlying properties of a system change.
This problem thus has received a lot of attention in the fields of data mining, etc... *** list and refs ***.
Often this problem is formulated in a statistical framework, by inspecting the data generating PDF (Probability Density Function) of the time series data.
A change point is then defined as a significant change in the properties of the PDF, such as the mean and variance.

*** Remove notion of GLR ***


The widely used CUSUM (cumulative sum) method by Basseville \etal \cite{basseville1993detection}, and the GLR (Generalized Likelihood Ratio) by Gustafsson \cite{gustafsson1996marginalized,gustafsson2000adaptive} take this approach.
The former originates from control methods for detection from bench marks.
The latter compares the logarithm of the likelihood ratio over two consecutive intervals.
These two methods is discussed and analyzed in section \ref{cusum}.

The GLR method, as with others, relies on pre-specified parametric model assumptions and considers the data be independent over time, which makes it less flexible to real-world applications.
The proposed methods by Kawahara \etal \cite{kawahara2009change} and Lui \etal \cite{liu2013change} try to overcome these problems by estimating the \emph{ratio} between the PDF, instead of estimating each PDF.
This approach is discussed and analyzed in section \ref{density-ratio}.

The density-estimation methods, as with the GLR, rely on the log-likelihood ratio between PDFs.
The method of Camci \cite{camci2010change} takes an other approach within the statistical framework, by using a SVM (Support Vector Machine).
One problem it tries to overcome is the (claimed) weakness of many methods to detect a decrease in variance.
The method represents the distribution over the data points as a hyper-sphere in a higher dimension using kernel trick.
A change in the PDF is represented by a change in the radius of this sphere.
Section \ref{svm} discusses the SVM-method.

% The final method under consideration, change detection via (intrinsic) dimensionality reduction, takes a different point of view.
% Opposed to the other discussed methods, dimensionality reduction is framed in the MDL (Minimum Description Length) framework.
% It uses the estimated underlying number of parameters of the time series as a model for change detection.
% Section \ref{dim-reduction} discuses this method.



%----------------------------------------------------------------------------------------
% CUSUM and GLR
% !TEX root = ../../main.tex
\section{CUSUM}\label{sec:cusum}

% \TODO{read book Basseville \cite{basseville1993detection}}


% Papers:
% \begin{itemize}
%   \item Use of Cumulative Sums of Squares for Retrospective Detection of Changes of Variance, 1994, 162 refs \cite{inclan1994use} \\
%   Implemented this algorithm.
%   \item An adaptive CUSUM-based test for signal change detection, 2006, 15 refs. \cite{alippi2006adaptive}
%   \item The MOSUM of squares test for monitoring variance changes \cite{hsu2007mosum}
% \end{itemize}

\emph{Notes:}
Non-Bayesian change detection algorithm (\ie no prior distribution believe available for the change time).
The \gls{cusum} method is developed by Page \cite{page1954continuous} for the application of statistical quality control (it is also known as a control chart).
Primary for detection of mean shift.
The \gls{mosum} of squares test for monitoring variance changes \cite{hsu2007mosum}.
Use of Cumulative Sums of Squares for Retrospective Detection of Changes of Variance \cite{inclan1994use}

An often used approach in the statistical framework of change detection is the \gls{cusum} as introduced by Page \cite{page1954continuous}.
Originally used for quality control in production environments, its main function is to detect change in the mean of measurements and has been applied to this problem \cite{basseville1993detection}.
It is a non-Bayesian method and thus makes no assumptions, in the form of prior belief distributions, for the change points.
Many extensions to this method have been proposed.
Some focus on the change in mean, such the method of Alippi and Roveri \cite{alippi2006adaptive}.
Others apply the method the problems in which the change of variance is under consideration.
Among others are there the centered version of the cumulative sums, introduced by Brown, Durbin and Evans \cite{brown1975techniques} and the \gls{mosum} of squares by \cite{hsu2007mosum}.

The method of Incl\'{a}n and Tiao \cite{inclan1994use} builds on the centered version of \gls{cusum} \cite{brown1975techniques} to detect changes in variance.
Using the batch \gls{icss} algorithm they are able to find multiple change points, offline while post-processing the data.
Let $C_k = \sum_{i=1}^k \alpha_t^2$ be the cumulative sum of squares for a series of uncorrelated random variables $\{\alpha_t\}$ of length $T$.
The centered (and normalized) sum of squares is defined as
\begin{equation}
  \begin{aligned}
  D_k = \frac{C_k}{C_T} - \frac{k}{T}, & & k = 1, \dots, T, & & \text{with } D_0 = D_T = 0.
  \end{aligned}
\end{equation}
For a series with homogeneous variance, the value of $D_k$ will oscillate around $0$.
In case of a sudden change, the value will increase and exceed some predefined boundary with high probability.
By using an iterative algorithm, the method is able to minimize the masking effect of successive change points.

One of the motivations for the \gls{icss} algorithm was the heavy computational burden involved with Bayesian methods, which need to calculate the posterior odds for the log-likelihood ratio testing.
The \gls{icss} algorithm avoids applying a function at all possible locations, due to the iterative search.
The authors recommend the algorithm for analysis of long sequences.

%----------------------------------------------------------------------------------------
% Density Ratio Estimation
% !TEX root = ../../main.tex
\section{Change-detection by Density-Ratio Estimation}\label{sec:density-ratio}

% Papers:
% \begin{itemize}
%   \item Change-Point Detection in Time-Series Data by Direct Density-Ratio Estimation, 2009, 45 refs \cite{kawahara2009change}
%   \item Change-Point Detection in Time-Series Data by Relative Density-Ratio Estimation, 2013, 3 refs \cite{liu2013change}
%   \item Density ratio estimation in machine learning (Book), 2009, 24 refs \cite{sugiyama2012density}
%   \item Direct importance estimation for covariate shift adaptation, 2008, 84 refs \cite{sugiyama2008direct}
% \end{itemize}

% Formulate the problem of detecting change in the statistical framework.
% Consider the probability distributions from which two consecutive segments of time series around a target time point are generated.
% When the disitrubtions differ significantly the target time point is regarded as a change point.




% CUSUM (cumulative sum) \cite{basseville1993detection} and GLR (generalized likelihood ratio)


% The distribution over the values of time series data can be represented with a probability density function (pdf).
% Two sections of a time series data can be generated with the same underlying pdf or each with a different.

Many approaches to detect change points monitor the logarithm of the likelihood ratio between two consecutive intervals.
A change point is regarded to be the moment in time when the underlying probabilistic generation function changes.
Some methods which rely on this are novelty detection, maximum-likelihood estimation and online learning of autoregressive models \cite{kawahara2009change}.
A limitation of these methods is that they rely on pre-specified parametric models.
Nonparametric models, for which the number and nature of the parameters are undetermined, for density estimation have been proposed, but it is said to be a hard problem \cite{hardle2004nonparametric, sugiyama2012density}.
A solution to this is to estimate the \emph{ratio} of probabilities instead of the probabilities themselves.
One of the recent methods to achieve this is the \gls{kliep} by Sugiyama \etal \cite{sugiyama2008direct}.

The proposed method by Kawahara and Sugiyama~\cite{kawahara2009change} uses an online version of the \gls{kliep} algorithm.
It considers \emph{sequences} of samples (rather than samples directly) because the time series samples are generally not independent over time.
The method detects change by monitoring the logarithm of the likelihood ratio between densities of reference (past) and test (current) time intervals.
If it exceeds a predetermined threshold value, the beginning of the test interval is marked as a change point.

Since the density ratio is unknown, it needs to be estimated.
The naive approach is to estimate it using estimated densities of the two intervals.
Since this is known to be a hard problem and sensitive for errors, the solution would be to estimate the ratio directly.

The method by Liu \etal \cite{liu2013change} estimates the ratio of probabilities directly, instead of estimating the densities explicitly.
Other methods using ratio-estimation are the Unconstrained Least-Squares Importance Fitting (uLSIF) method \cite{kanamori2009least}, and an extension which possesses a superior nonparametric convergence property: Relative uLSIF (RuLSIF) \cite{yamada2013relative}.

%----------------------------------------------------------------------------------------
% Support Vector Machines
% !TEX root = ../../main.tex
\section{Change-detection by Support Vector Machines}\label{svm}

Introduced by Vapnik \cite{vapnik1998statistical, vapnik1999nature}, Support Vector Machines offer a way to segment, and classify, linear separable data.
This technique can also be applied to estimate density functions of given time series \cite{weston1999support}.
When combined with a mapping function, which maps the data from the input space $I$ to a higher dimension feature space $F$, the input data can be non-linear separable.
The linear hyperplane, which segments the data in the feature space $F$, yields to a non-linear segmentation in the lower-dimensional input space $I$.
Instead of explicitly mapping the input data to the higher dimensional space, a kernel function can be used.
This kernel function can calculate values of the feature space directly, without the need to first map the input values to this space.
This process is referred to as the kernel trick.

The method of Camci \cite{camci2010change} uses one-class support vector machines to segment time series data.
For the current interval a hyper-sphere is constructed, using the kernel trick, which circumscribes all the data.
To cope with possible errors or outliers a soft-margin is applied.
The data can thus be represented using a center $c_1$ and radius $r_1$ of the hyper-sphere.
New data points are consecutively mapped from the input space to the feature space.
The retrieved (high dimension) data point can be in- or outside the earlier constructed hyper-sphere, thereby giving information about a possible change point.



\subsection{One-class Support Vector Machine}
*** This section is too detailed for the Literature Review, but keep it for now ***


*** Replace with / add information from summary in \ref{svm-explained}. ***

The proposed method of Camci \cite{camci2010change} uses a one-class support vector machine to segment time series data.
One-class SVMs are used to describe the current data under consideration, by assuming all data points are from the same class \cite{tax2001one}.
The class is described by a spherical boundary around the data with center $c$ and radius $r$, such that the volume is minimized.
Following the definition of Camci \cite{camci2010change}, the class description is obtained by minimizing $r^2$:
\begin{equation}
  \mathrm{Min}\ r^2
\end{equation}
\begin{equation}
  \mathrm{Subject\ to} : \|\mathbf{x}_i - \mathbf{c}\|^2 \le r^2\ \forall i,\ \mathbf{x}_i : i \mathrm{th\ data\ point}
\end{equation}

To be able to handle outliers in the input data, a penalty cost function $\varepsilon_i$ for each outlier can be added.

*** Add new function and constraints? ***

Using this one-class SVM formulation, differences between two (consecutive) windows of data points with size $w$ can be obtained.
The first window is used as the input set, $h_1$ and the second as the test set $h_t$.
For the first window a one-class SVM is constructed, yielding in a representation by $c_1$ and $r_1$.
When the data points of the second window belong to the same class, the representation of that one-class SVM would equal the first:
\begin{equation}
\begin{aligned}
  c_1 = c_2, & &  r_1 = r_2
\end{aligned}
\end{equation}

*** First tell more about (underlying) probability density functions, to relate to other methods ***

In case the second window of data points does not belong to the same class, i.e. the probability density function that describes the data differs from the first, the describing values of the second window will differ from the first.
The amount of difference can be expressed by a dissimilarity measure over the representations.
When the dissimilarity between the two windows exceeds some predefined threshold $th$, there exists a change point between the windows.

This process can be visualized as done in *** insert figure of four circles ***. The second window, $h_2$ can be constructed from the first by e.g. a shift of one data point. *** explain data point positions by circle ***.

Note that a difference in the SVM center $c$ or radius $r$ represent a change in the mean and variance, respectively.



%----------------------------------------------------------------------------------------
% Dimensionality reduciton
% % !TEX root = ../../main.tex
\section{Change-detection by Dimensionality Reduction / Covariance structure}\label{dim-reduction}


