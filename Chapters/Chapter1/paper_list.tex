% !TEX root = ../../main.tex

*****
\subsection{List of papers}
List of all papers, shortly categorized
\begin{itemize}
  \item ``Discovering characteristic actions from on-body sensor data'' \cite{minnen2006discovering}, cited: 63. Motifs, HHM, DTW, 87\% accuracy. Discovers motif (actions) from data. Worn on wrist, single. Environment artificial.
  \item ``Recognition of human activities using layered hidden Markov models'' \cite{perdikis2008recognition}, cited: 3. HMM, layerd (primitive and abstract actions). Uses vision for workplace-activities. Direct classification. No sensors. Artificial environement for region labeleing.
  \item ``Accelerometer-Based Gait Analysis, A survey'', \cite{derawi2010accelerometer}, cited: 4. Compares methods to distinguish walking of normal, fast, slow. Focus on gait, but compares classification methods such as SVM, PCA, KSOM.
  \item ``An Automatic Segmentation Technique in Body Sensor Networks based on Signal Energy'', \cite{guenterberg2009automatic}, cited: 13. Automatic segmenting, adaptive threshold. Pure segmenting, no classification. Nice and clean method? Weakness: single action segmented as multiple. Used multiple sensors.
  \item ``Towards HMM based Human Motion Recognition using MEMS Inertial Sensors'', \cite{shi2009towards}, cited: 13. Uses HHM, Fourier transform for features. 5 fixed activities, trained HMM. Correct rates from 90\% to 100\%. Classification, not just segmentation. Single sensor.
  \item ``Change-Point Detection in Time-Series Data by Direct Density-Ratio Estimation'', \cite{kawahara2009change}, cited: 41. Non-parametric approach, online method, no strong model assumptions. Uses multiple datasets, origin referenced. Much follow-up work done. Artificial datasets. (no sensors)
  \item ``Unsupervised, Dynamic Identification of Physiological and Activity Context in Wearable Computing'', \cite{krause2003unsupervised}, cited: 106(!). Combination of KSOM, iterative K-means clustering, transient states removal (using markov model). On-line algorithm. Single sensor on upper arm. Tested for a few days and controlled environment.
  \item ``Activity Recognition using Cell Phone Accelerometers'', \cite{kwapisz2011activity}, cited: 122 (!), 2011. Uses WEKA: decision trees, logistic regression and multilayer neural networks. Results up to 98\% (jogging). Much difficulty with stairs up and down. Single accelerometer, front leg pocket. Offline method. Many recent references. Controlled environment.
  \item ``Activity recognition from accelerometer data'', \cite{ravi2005activity}, cited: 434 (!), 2005. Compares 18 (offline) classifiers, base level (WEKA) and meta-level. Single device, pelvic region. Controlled setting.
  \item ``Activity recognition from user-annotated acceleration data'', \cite{bao2004activity}, cited: 1008 (!!), 2004. Five bi-axial sensors. Decision tables, instance-based learning, C4.5 (trees, highest accuracy 89.30\%) and naive bayes (weka), twenty activities. Controlled environment, common room.
  \item ``Using Hierarchical Clustering Methods to Classify Motor Activities of COPD Patients from Wearable Sensor Data'', \cite{sherril2005using}, cited: none, 2005. Uses Linear Discriminant Analysis for cluster classification (using Cluster Quality Index to determine k) and simple rule-based separation for high level ambulatory. Hierarchical Dendogram to merge clusters when similarity to high. Sensors on forearms and legs. Controlled clinical environment.
  \item ``Non-Parametric Bayesian Human Motion Recognition Using a Single MEMS Tri-Axial Accelerometer'', \cite{ahmed2012non}, cited: none, 2012. Recognizes the number of human activities, single sensor on the chest. No training data. Infinite Gaussian Mixture Model, collapsed Gibbs sampler. Compares with parametric Fuzzy C-mean (data point belongs to multiple clusters), unsupervised K-means, non-parametric mean-shift. Outperforms all significantly, high hit rate and low false alarms.
  \item ``An Online Algorithm for Segmenting Time Series'', \cite{keogh2001online}, cited: 487, 2001. Reviews algorithms to get a piecewise linear representation, proposes a novel sliding-window and bottom up approach, on-line. Mere segmentation. No classification; only approximation of signal. (Dataset available).
  \item ``Segmenting Motion Capture Data into Distinct Behaviors'', \cite{barbivc2004segmenting}, cited: 242, 2004. On-line methods: cut-points on increased intrinsic dimensionality, distribution of poses is observed to change. Batch: cut when consecutive frames belong to different Gaussian mixture models. Fixed fourteen motions, compared with manual segmentation. Uses PCA and Probabilistic PCA (models the non-pc subspaces as noise). PPCA is the best. (paper anne)
  \item ``Aligned Cluster Analysis for Temporal Segmentation of Human Motion'', \cite{zhou2008aligned}, cited: 54, 2008. Uses extension on kernel k-means clustering and Dynamic Time Alignment Kernel (kernel of DTW) for temporal invariance, robust temporal matching metric. Coordinate descent algorithm solves ACA. Fixed number of clusters, trapped in local minima. (paper anne). Uses motion capture data.
  \item ``Bayesian Nonparametric Methods for Learning Markov Switching Processes'', \cite{fox2010bayesian}, cited: 10, 2010. Uses HMM for state-space model for segmentation, Markov Jump Linear systems. Unbounded number of Markov modes (parameters). Number of models is fixed, though? (paper anne). No sensors.
  \item ``Layered Representations for Human Activity Recognition'', \cite{oliver2002layered}, cited: 215, 2002. Layers of Hidden Markov Models HMM, multiple levels of granularity (Based on intuition) and context. Less retraining, only lower models. Classified also. No sensors, visual and akoustic data in office surroundings.
  \item ``Time series segmentation for context recognition in mobile devices'', \cite{himberg2001time}, cited: 151, 2001. Splits and merges segments, while keeping k constant using cost function on segments for internal heterogeneity. Multiple instances determine number of k. Microphone and accelerometer data, in front of chest. ARtificial data
  \item ``A practical approach to recognizing physical activities'', \cite{lester2006practical}, cited: 259, 2006. Uses method of \cite{lester2005hybrid}, activity classification algorithm. Selects most useful features and then recognizes walking, sitting, etc. First layer static classifier on features or data (energy, mean, variance, correlation, etc), then a layer of HMM to estimate activity. All offline. Shows trained model is robust to location of wearing on body.
  \item ``A hybrid discriminative/generative approach for modeling human activities'', \cite{lester2005hybrid}, cited: 252, 2005. Used in method above. Combines boosting to select and reduce useful features and learn static classifiers with HMM to capture regularities and smooth activities (quick switching is unlikely). Single sensor, multiple measures (accelerometer, audio, etc), worn on the shoulder.
  \item ``Using acceleration measurements for activity recognition: An effective learning algorithm for constructing neural classifiers'', \cite{yang2008using}, cited: 67, 2008. Separates dynamic from static activities. Uses multilayer feedforward neural networks to generate complex discriminating surfaces as activity classifiers. Feature subset selection approach is developed. Neural pre-classifier with constant threshold criterion. Uses Common Principal Component Analysis. Recognizes eight activities with 95\% overall accuracy. Laboratory environment. Worn on dominant wrist.
  \item ``Single-accelerometer-based daily physical activity classification'', \cite{long2009single}, cited: 46, 2009. Uses Naive Bayes classifier, sensor worn on wrist. Five activities. PCA to reduce dimensions and create independence for NB. Comparable results as Decision Trees, but with flexibility to add activities. Performed outside laboratory, 10 hours long.
  \item ``Distributed Continuous Action Recognition Using a Hidden Markov Model in Body Sensor Networks'', \cite{guenterberg2009distributed} cited: 17, 2009. Single HMM, sensor network cluster movements, HMM constructs continuous actions using postures and actions, merges left-to-right HMM for each action to a single. Transcript generation per sensor uses Gaussian Mixed Model, multiple models with $m$ mixtures are trained with Expectation-Maximization (EM), best is chosen.
  \item ``Offline and online activity recognition on mobile devices using accelerometer data'', \cite{duque2012offline}, cited: none, 2012. Compares offline and online methods, cellphone in pocket. Reduces feature extraction (mean and standard deviation (for resources of cellphone). Training is done by database and labeling, classification with K-nearest neighbors, decision trees (C4.5) and decision rules. Online and offline KNN-3 is best, 99\% and 97\%. Six daily activities. Trousers pocket.
  \item ``Recognizing human activities user-independently on smartphones based on accelerometer data'', \cite{siirtola2012recognizing}, cited: none, 2012. Classifiers K-nearest neighbors and QDA (quadratic discriminant analysis). Online regocnition implemented on phone. five daily activities. Model training by decision tree (active/inactive). Results for QDA, online: 95\%. Activities performed outside. Trousers pocket.
  \item ``Activity recognition from acceleration data based on discrete consine transform and SVM'', \cite{he2009activity}, cited: 25, 2009. Uses Discrete Cosine Transform (DCT) to get features, PCA to reduce them an multiple SVMs to classify windows. Four daily activities, 97\% precision. Trousers pocket. Laboratory setting.
  \item ``Physical Activity Recognition Using a Single Tri-Axis Accelerometer'', \cite{lee2178physical}, cited: 16, 2009. Five daily activities, FFT, fuzzy c means classification, offline. 99\% accuracy. No smartphone, other device. Outside, although restricted (standing, sitting, lying, walking, running)
\end{itemize}