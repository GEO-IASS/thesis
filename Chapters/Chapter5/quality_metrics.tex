% !TEX root = ../../main.tex
\section{Quality metrics}\label{sec:artificial_data_quality_metrics}
For an objective measure of quality of the proposed method applied to the four aforementioned data sets, we employ two metrics, inspired by and slighty altered from Takeuchi and Yamanishi~\cite{takeuchi2006unifying}.
The first metric is the $\operatorname*{average \_ benefit}$, for which we first need to define the notion of $\operatorname*{benefit}$ for each change:
\begin{equation}\label{eq:benefit}
  \operatorname*{benefit}(y) = 1 - |t_y - t_y^*|,
\end{equation}
where $t_y$ is the time of the closest discovered change point and $t_y^*$ is the true change time.
We can then define the $\operatorname*{average \_ benefit}$ over all the discovered change points $Y$ as:
\begin{equation}\label{eq:average_benefit}
  \operatorname*{average \_ benefit}(Y) = \frac{\sum_{y=1}^Y \operatorname*{benefit}(y)}{Y}.
\end{equation}
Besides the benefit, which rewards discovered change points close to the actual change points, we need to penalize discovered change points which do not relate to a true change point.
For this we use the \acrlong{far}, expressed as
\begin{equation}\label{eq:false_alarm_rate}
  \operatorname*{far}(Y) = \frac{Y_{\operatorname*{false}}}{Y}.
\end{equation}

For a high quality method the $\operatorname*{average\_benefit}$ should be high and the $\operatorname*{far}(Y)$ should be low.
In our results we will also list the parameters used, since these need to be user-defined and are part of the (manual) optimization process.
The parameters are:
\begin{itemize}
  \item \textbf{Window length}: The window length that is processed by the \gls{svdd} algorithm. This is the number of data points that are continuously modeled in the shape in a hypersphere.
  \item \textbf{Sigma of \gls{rbf}}: The value of $\sigma$ for the \gls{rbf} kernel. The effect of this parameter is discussed in \Cref{subsec:svm_model_parameters}.
  \item \textbf{High threshold}: The $th_\text{high}$ threshold value for the \gls{rt} method described in \Cref{subsec:ratio_thresholding}. A higher value indicates a more precise and robust setting.
  \item \textbf{Low threshold}: The $th_\text{low}$ threshold value for the \gls{rt} method described in \Cref{subsec:ratio_thresholding}. A lower value indicates a more precise and robust setting.
  \item \textbf{Closeness}: In the post-processing phase the detected change points which are within a certain \emph{closeness} range will be merged together. This parameter indicates the minimal number of data points that change points must differ in order to be considered separate detections.
\end{itemize}
For all the sets we have used a value of $C = 0.1$, which indicated that $10\%$ of the data points are used as outliers.
This is a normal setting for the \gls{svdd} method.