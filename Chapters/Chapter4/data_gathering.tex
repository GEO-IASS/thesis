% !TEX root = ../../main.tex
\section{Data Gathering}\label{sec:method_data_gathering}
In this section we briefly discuss the different data gathering methods used for the change detection algorithms and experiments.
Section \ref{subsec:data_gathering_artificial} reviews the artificial data sets we will use.
In Section \ref{subsec:data_gathering_real_world} an overview of the real-world data sets used is provided.
Both sections refer to Chapters \ref{Chapter5} and \ref{Chapter6} for more details, respectively.

% ------
\subsection{Artificial data}\label{subsec:data_gathering_artificial}
In order to provide an objective comparison to other methods, we will use artificial data sets which are also used in the researches on which our method is based.
The data sets are from Takeuchi and Yamanishi \cite{takeuchi2006unifying} and Camci \cite{camci2010change}.
Both construct a collection of one-dimensional time series data according to the second order \gls{ar} model:
\begin{equation}
  x_t = a_1 x_{t-1} + a_2 x_{t-2} + \epsilon_t.
\end{equation}
In the different data series the mean and variance of the Gaussian random variable $\epsilon_t$ differs and changes at pre-determined change points.
Using this data set an objective quality measure over the change detection methods can be obtained and compared.
All the used data sets are listed and analyzed in Chapter \ref{Chapter5}.

% ------
\subsection{Real-world data}\label{subsec:data_gathering_real_world}
In the second type of data sets we apply our method of change detection and temporal segmentation to real-world data sets.
Our setup records the activities of humans performed both in and out door in an uncontrolled environment.
Activities performed include walking, running in both a straight line and a curve, standing, walking up and downstairs and sitting.
Using the time series data sensor output in the form of acceleration, gravity and rotation metrics, our method gives change points based on those signals.
By comparing the discovered change points with annotated change points (based on video recordings of the performed activities) we are able to give subjective results.
In Chapter \ref{Chapter6} we give a detailed analysis of the performed activities and the recorded data sets.