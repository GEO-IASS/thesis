% !TEX root = ../../main.tex
\section{Data Gathering}\label{sec:method_data_gathering}
In this section we briefly discuss the different data gathering methods used for the change detection algorithms and experiments.
\Cref{subsec:data_gathering_artificial} discusses the artificial data sets we will use.
In \Cref{subsec:data_gathering_real_world} an overview of the real-world data sets used is provided.
Both sections refer to Chapters~\ref{Chapter5} and~\ref{Chapter6} for more details, respectively.

% ------
\subsection{Artificial data}\label{subsec:data_gathering_artificial}
In order to provide an objective, quantitative, comparison to other methods, we will use artificial data sets which are also used in the earlier work on which \gls{ocs-hats} is based.
These are the data sets used by Takeuchi and Yamanishi \cite{takeuchi2006unifying} and Camci \cite{camci2010change}.
Both construct a collection of one-dimensional time series data according to a second order \gls{ar} model:
\begin{equation}
  x_t = a_1 x_{t-1} + a_2 x_{t-2} + \epsilon_t.
\end{equation}
Over the different data series the mean and variance of the Gaussian random variable $\epsilon_t$ differs and changes at pre-determined change points.
Using this data set an quantitative quality measure over the change detection methods can be obtained and compared.
All the used data sets are listed and analyzed in \Cref{Chapter5}.

% ------
\subsection{Real-world data}\label{subsec:data_gathering_real_world}
In the second type of data sets we apply our method for change detection and temporal segmentation to real-world data sets.
We record the activities of humans in real-world environments, both in- and outdoor.
Activities performed include sitting, standing, walking, running in a straight and curved line, and walking up- and downstairs.
\gls{ocs-hats} uses the signals from the accelerometer, magnetic field, and rotation sensors.
These time series data are used to detect change points.
A video recording from the performed activity is used to annotate the time series with real change points.
The discovered change points are compared with these annotated change points to give a qualitative quality measure.
In \Cref{Chapter6} we give a detailed analysis of the performed activities and the recorded data sets.

For the experiments we used a smartphone with inertial sensors as recording device.
The activities were recorded using a free \textsc{Android} application \textsc{Sensor Logger}\footnote{https://play.google.com/store/apps/details?id=
com.kzs6502.sensorlogger}.
This application was chosen for its convenient data format of the sensor recording and its regularity of the sampling interval.
Table~\ref{tab:recorded_metrics} lists all the recorded metrics.
For our experiments we used the data for the accelerometer, magnetic field and rotation.

We implemented our algorithm in \textsc{Matlab}.
For the \gls{svdd} model construction phase we used the \textsc{dd\_tools} library by Tax \cite{Ddtools2013}, which depends on the \textsc{PR\_tools} package, originating from the book by Van Der Heijden \etal \cite{van2005classification}.
Alternatively, the widely used \textsc{LibSVM} \cite{chang2011libsvm} add-on for the \gls{svdd} algorithm by Chang \etal \cite{changrevisit} can be used.

\begin{center}\begin{table}
  \caption[Measured metrics]{Measured sensor metrics. The set of axis is always the triple (x, y, z) direction.}
  \begin{tabulary}{\textwidth}{|l|L|c|c|}
    \hline
    Sensor metric & Description & Units of measure & Typical range \\
    \hline \hline
    Accelerometer & Acceleration force along each axis (including gravity). & $m/s^2$ & $-20$ -- $20$ \\
    \hline
    Gravity & Force of gravity along each axis. & $m/s^2$ & $-10$ -- $10$\\
    \hline
    Gyroscope & Rate of rotation around each axis. & $rad/s$ & $-15$ -- $15$\\
    \hline
    Light & Light sensitive sensor at the front of the phone. & & $0$ -- $10000$ \\
    \hline
    Linear acceleration & Acceleration force along each axis (excluding gravity). & $m/s^2$ & $-20$ -- $20$ \\
    \hline
    Magnetic field & Geomagnetic field strength along each axis. & $\mu T$ & $-60$ -- $60$ \\
    \hline
    Orientation & Degrees of rotation around the three physical axis. & Degrees & $-100$ -- $360$ \\
    \hline
    Rotation & Measure of rotation around the device's rotation axis. & Unitless & $-1$ -- $1$\\
    \hline
  \end{tabulary}

  \label{tab:recorded_metrics}
\end{table}\end{center}
