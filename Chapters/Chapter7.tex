% !TEX root = ../main.tex
% Chapter 7

\chapter{Conclusion}

\label{Chapter7} % For referencing the chapter elsewhere, use~\ref{Chapter7}

\lhead{Chapter 7. \emph{Conclusion}} % This is for the header on each page - perhaps a shortened title

%----------------------------------------------------------------------------------------
\section{Outline}
\emph{Not intended for the reader.}
\begin{itemize}
  \item Tie together, integrate, synthesize all the issues from the chapters, reflecting on introductory thesis statements.
  \item Provide answers to research questions
  \item Highlight study limitations
  \item Directions for further research.
  \item Content: Summarize
  \begin{itemize}
    \item What have I researched
    \item Main arguments
    \item How researched
    \item What discovered
    \item What pre-existing views were challenged (?)
  \end{itemize}
  \item Overview of
  \begin{itemize}
    \item New knowledge
    \item Significance of research
    \item Limitations (data, concepts)
    \item Speculations on implications of research
    \item Further research and development, or: links with different fields, other methods applied to same data.
  \end{itemize}
  \item Must: make clear statement of my original contribution
  \item Ideally:
  \begin{itemize}
    \item Shows links across key ideas spread in chapters
    \item Show commitment and enthusiasm for research
    \item Positive impression
  \end{itemize}
  \item Avoid:
  \begin{itemize}
    \item claim findings that have not been proven
    \item introducing new data/concepts
    \item Be self-critical, but dont put weaknesses or limitation in research
    \item Too long/too short
  \end{itemize}
\end{itemize}


Paragraph: into on conclusion; what have I researched and what is my contribution.
Break into sections further down.

===

In this research we have applied \gls{occ} methods to accelerometer data, recorded during the performance of various human activities, in order to create a temporal segmentation of those recordings.
The goal was to find the change points between activities, assuming that it will aid better understanding of the sensor values.
To do so, we used the \gls{svdd} algorithm by Tax~\cite{tax1999support}, following the method of applying it to a sliding window of data, as done by Camci~\cite{camci2010change}.
Instead of using both the information whether a new data point is an outlier and the radius of the hypersphere to indicate change, we only used on the latter model properties.
The increase of the radius of the constructed hypersphere, which encloses the data objects and increases with heterogeneous segments of data, is extracted and used for indication of change.

\TODO{Write overview of this chapter. \emph{In the following section we will...}}


\section{What is done}
The scope of this research was the temporal segmentation of recordings, obtained from human activities.
This was performed in the wider context of activity classification.
Currently, many algorithms obtain an implicit segmentation as a side-effect of direct activity classification.
In this research the primary goal was finding a temporal segmentation, which is assumed to be able to aid the classification step.

We have considered a range of segmentation methods, from which the method by Camci~\cite{camci2010change} showed potential.
It is based on the detection of outliers in a time series data, assuming that an increase of outliers indicates a change in the underlying generating model.
For the detection of outliers it uses the \gls{svdd} method by Tax~\cite{tax1999support}.
It tests every new data point with the constructed model; whether the data point is in or outside the model, and whether the model increases or decreases in radius size.
The constructed sliding-window algorithm, \gls{svcpd}, is applied to artificial Gaussian noise data.

In this research we have applied the method by Camci~\cite{camci2010change} to inertial signals, recorded by smartphones.
We have used the accelerometer data for measurement of speed, the gyroscope data for measurement of rotation, and the magnetometer to measure the direction and orientation of the performed activities.
The constructed method reduces this 9-dimensional signal to a single property, obtained from the constructed model, which can be interpreted to indicate change.

The constructed model follows the implementation of Tax and Laskov~\cite{tax2003online} for an incremental version of \gls{svdd}, \gls{isvdd}.
This algorithm constructs a \gls{oc-svm} model by processing the data over a sliding window.
The model represents a hypersphere from which the radius size is abstracted.
That property is of interest since the assumption is that a heterogeneous window of data will show an increase of radius, in relation to a homogeneous segment of data.
The homogeneous segment of data will have a relatively small radius since the data points will be close together, resulting from the continuity assumptions.

To test our approach, we have used the same data as used in the experiments from Camci~\cite{camci2010change} and Takeuchi and Yamanishi~\cite{takeuchi2006unifying}.
Furthermore, since this research is focused around human activities, we have recorded and manually annotated in- and outdoor activities.
The discovered change points were compared to the annotated change points, obtained from video recordings.
In this research we found that other public available and wide used common data sets, such as the WISDM~\cite{kwapisz2011activity} and UCI HAR~\cite{anguita2012human}, were not useful for our purpose, since the activities are non-continuously recorded.



\TODO{In the above, refer to the corresponding chapters/sections.}

\section{Main findings}
What we have found, show links across chapters.

For each finding a different paragraph.
How did we arrive at this finding, and it challenges/improves on previous research.
Not just a summary, but the effects and implications.

While temporal segmentation of time series data is not uncommon, the application to inertial signals is currently not under much research attention.
It is applied in other contexts, such as motion tracking and first person vision sensing, the segmentation obtained from inertial signals is often the implicit result of classification.

\section{Contributions}


\section{Other research}
Relate the finding with earlier research.
Provide similarities and differences.
Better/worse results?

\section{Limitations}
Show what limitations were encountered or decided (data collection, scope).
Note the focus of the limitations and how they worked together (to get positive note).
This can create bridge to "future research".

\section{Future research}
The motivation for this research was, amonst others, the assumption that an explicit temporal segmentation of inertial signals can be used to obtain a better classification of the performed activities.
There are two reasons for that.
First, since the segmentation adds information about sequential data points (whether they belong to the same class), the a priori knowledge about the distribution changes.
Second, the window of data used for model construction can be much larger (than the often used length of 1-3 seconds).
This enables a better mathematical model to be constructed.
A future research in this line of assumptions is to compare the performance of classification methods using this obtained temporal segmentation and those that do not.

Since the temporal segmentation of inertial signal is not yet a very active field of study, more studies can be performed to obtain it.
In our approach we have used the raw data directly.
In the field of activity recognition many methods make use of pre-processing steps, from which one step often is feature extraction from the raw data.
That, and other pre-processing steps, could be incorporated in the explicit temporal segmentation of inertial signals.
Furture research could focus on extracted features that aid the detection of change points.

\TODO{Add future research on \gls{oc-svm} applications.}


What has not been covered? What is worthwhile.
Show that I am thinking ahead and know the field of study.



\section{Final section}
Reminding of original contribution and significance.
This section is the overall conclusion and thus should be very concise and precise.

\TODO{Formulate the research questions in the introduction of thesis, and in the chapters.}