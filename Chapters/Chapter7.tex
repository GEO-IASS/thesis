% !TEX root = ../main.tex
% Chapter 7

\chapter{Conclusion}

\label{Chapter7} % For referencing the chapter elsewhere, use~\ref{Chapter7}

\lhead{Chapter 7. \emph{Conclusion}} % This is for the header on each page - perhaps a shortened title

%----------------------------------------------------------------------------------------
\section{Outline}
\emph{Not intented for the reader.}
\begin{itemize}
  \item Tie together, integrate, sythesize all the issues from the chapters, reflecting on introductory thesis statements.
  \item Provide answers to research questions
  \item Highlight study limitations
  \item Directions for further research.
  \item Content: Summarize
  \begin{itemize}
    \item What have I researched
    \item Main arguments
    \item How researched
    \item What discovered
    \item What pre-existing views were challenged (?)
  \end{itemize}
  \item Overview of
  \begin{itemize}
    \item New knowledge
    \item Significance of research
    \item Limitations (data, concepts)
    \item Speculations on implications of research
    \item Further research and development, or: links with different fields, other methods applied to same data.
  \end{itemize}
  \item Must: make clear statement of my original contribution
  \item Ideally:
  \begin{itemize}
    \item Shows links across key ideas spread in chapters
    \item Show commitment and enthusiasm for research
    \item Positive impression
  \end{itemize}
  \item Avoid:
  \begin{itemize}
    \item claim findings that have not been proven
    \item introducing new data/concepts
    \item Be self-critical, but dont put weaknesses or limitation in research
    \item Too long/too short
  \end{itemize}
\end{itemize}


Paragraph: into on conclusion; what have I researched and what is my contribution.
Break into sections further down.

===

In this research we have applied \gls{occ} methods to accelerometer data, recorded during the performance of various human activities, in order to create a temporal segmentation of those recordings.
The goal was to find the change points between activities, assuming that it will aid better understanding of the sensor values.
To do so, we used the \gls{svdd} algorithm by Tax~\cite{tax1999support}, following the method of applying it to artificial data as done by Camci~\cite{camci2010change}.
Instead of using both the information whether a new data point is an outlier and the radius of the hypersphere to indicate change, we only used on the latter model properties.
The increase of the radius of the constructed hypersphere, which encloses the data objects through a sliding window, is extracted and used for indication of change.

In the following sections we will...


\section{What is done}
What we have researched and how we did it.
Repeat research question, justify the research.
Provide context, background of the topic.
Indicate the problem (which links to research question and key objectives).
Explain what is further discussed in the conclusion (sequential map of subjects).

\section{Main findings}
What we have found, show links across chapters.

For each finding a different paragraph.
How did we arrive at this finding, and it challanges/improves on previous research.
Not just a summary, but the effects and implications.

\section{Other research}
Relate the finding with earlier research.
Provide similarities and differences.
Better/worse results?

\section{Limitations}
Show what limitations were encountered or decided (data collection, scope).
Note the focus of the limitations and how they worked together (to get positive note).
This can create bridge to "future research".

\section{Future research}
What has not been covered? What is worthwhile.
Show that I am thinking ahead and know the field of study.

\section{Final section}
Reminding of original contribution and significance.
This section is the overall conclusion and thus should be very concise and precise.

\TODO{Formulate the research questions in the introduction of thesis, and in the chapters.}