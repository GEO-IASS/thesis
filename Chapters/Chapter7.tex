% !TEX root = ../main.tex
% Chapter 7

\chapter{Conclusion}

\label{Chapter7} % For referencing the chapter elsewhere, use~\ref{Chapter7}

\lhead{Chapter 7. \emph{Conclusion}} % This is for the header on each page - perhaps a shortened title

This research has applied \gls{occ} methods to inertial sensor data, recorded during the performance of various human activities, in order to create a temporal segmentation of those recordings.
The goal was to find the change points between activities, assuming that it will aid our understanding of the sensor values.
To do so, we used the \gls{svdd} algorithm by Tax~\cite{tax1999support} and applied it to a sliding window of data, as done by Camci~\cite{camci2010change}.
From these methods we have created our own method \gls{ocs-hats} and made it publicly available \cite{vlasveld2014hats}, which is a slightly modified implementation of \gls{svcpd} by Camci~\cite{camci2010change} applied to \gls{har} data.
% Instead of using both the information whether a new data point is an outlier and the radius of the hypersphere to indicate change, we only used the latter model properties.
An increase of the radius of the constructed hypersphere, which encloses the data objects and increases with heterogeneous segments of data, is extracted and used for indication of change.

In \Cref{sec:proposed_method} we will further discuss the proposed method for temporal segmentation of inertial sensor signals from human activities.
Our main findings and contributions are discussed in \Cref{sec:main_findings}.
The limitations of our method and research is subject of \Cref{sec:limitations}.
\Cref{sec:further_research} provides suggestions for future research.
\Cref{sec:final_remarks} is the final section of this thesis.

\section{Proposed method}\label{sec:proposed_method}
The goal of this research was to find a temporal segmentation of recordings obtained from human activities.
This was performed in the wider context of activity classification.
Currently, many algorithms obtain an implicit segmentation as a side-effect of direct activity classification.
In this research the primary goal was finding a temporal segmentation, which is assumed to be able to aid the classification step.
This problem is reduced to finding change points in temporal data sets originating from inertial sensors.

We have considered a range of segmentation methods, from which the method by Camci~\cite{camci2010change} showed potential.
It is based on the detection of outliers in a time series data, assuming that an increase of outliers indicates a change in the underlying generating model.
For the detection of outliers it uses the \gls{svdd} method by Tax~\cite{tax1999support}.
It tests every new data point with the constructed model; whether the data point is in or outside the model, and whether the model increases or decreases in radius size.
The constructed sliding-window algorithm, \gls{svcpd}, is applied to artificial Gaussian noise data.

In this research we have applied the method by Camci~\cite{camci2010change} to data from inertial sensors, recorded by smartphones.
We have used the accelerometer data for measurement of speed, the gyroscope data for measurement of rotation, and the magnetometer to measure the direction and orientation of the performed activities.
The proposed method, as discussed in \Cref{Chapter4}, reduces a high dimensional signal to a single property, obtained from the constructed model, which can be interpreted to indicate change.

The constructed model follows the implementation of Tax and Laskov~\cite{tax2003online} for an incremental version of \gls{svdd}, \gls{isvdd}.
This algorithm creates a \gls{oc-svm} model by processing the data over a sliding window.
The model represents a hypersphere from which the radius size is extracted.
That property is of interest since the assumption is that a heterogeneous window of data will show an increase of radius, in relation to a homogeneous segment of data.
The homogeneous segment of data will have a relatively small radius since the data points will be close together, resulting from the continuity assumptions.

To test our approach, we have used the same data as used in the experiments from Camci~\cite{camci2010change} and Takeuchi and Yamanishi~\cite{takeuchi2006unifying}, for which the results are comparable with the original research (see \Cref{sec:artificial_data_results}).
Furthermore, since this research is focused around human activities, we have recorded and manually annotated in- and outdoor activities.
The discovered change points were compared to the annotated change points, obtained from video recordings.
The results where discussed in \Cref{sec:real_world_results}.
In this research we found that other public available and wide used common data sets, such as the WISDM~\cite{kwapisz2011activity} and UCI HAR~\cite{anguita2012human}, were not useful for our purpose, since the activities are non-continuously recorded.

\section{Contributions}\label{sec:main_findings}
While temporal segmentation of time series data is not uncommon, the application to inertial signals is currently not under much research attention.
For other contexts it has been applied, such as motion tracking \cite{barbivc2004segmenting,li2007segmentation}.
In the context of human activities the segmentation obtained from inertial sensor signals is often an implicit result of classification.

We have shown that it is possible to explicitly obtain a temporal segmentation, expressed as change points, of activities performed by humans.
This was performed in a real-world setting, showing the robustness of the method for noisy data.
Since manual labeling (the annotation of change points on the time series data) of human activities is ambiguous we put emphasis on the qualitative inspection of the results.
The proposed method gives fairly good results, although parameter tuning is a subject of further research.

We have constructed \gls{ocs-hats} as a slightly simplification of \gls{svcpd} by Camci~\cite{camci2010change} application to real-world inertial sensor data.
The \textsc{Matlab} implementation is free to use for the public \cite{vlasveld2014hats}.
We have tested our proposed method with the same artificial data and did not find a deterioration in the results.
Because of the noisy nature of the recorded activities and the ability to segment the time series, we have shown that \gls{ocs-hats} is a robust method for real-world applications.

The proposed method is, as far as we know, the first application of \gls{oc-svm} based methods to on-body worn inertial sensor signals, such as accelerometer data.
Furthermore, it applies the algorithms, amongst others of Tax and Duin~\cite{tax1999support}, to real-world data.
We have shown that the assumption of a change the performed activity results, due to a change in the data distribution, in an increase of the hypersphere's radius.
This increase, and a similar decrease in the case of homogeneous activity distribution in the current window, can be used for an indication of change.

The proposed \gls{ocs-hats} method processes the raw inertial sensor data, originating from sensors as accelerometers and magnetic field sensors.
This contrasts with many other approaches that extract features in a pre-processing stage.

Furthermore, because the it is an online method, it is suitable for applications that need direct segmentation of time series data.

The commonly used data sets \cite{kwapisz2011activity,anguita2012human} are recorded in a non-continuous manner.
The proposed method requires data streams of continuous activities.
Annotating change points on this kind of time series can be ambigious, so we needed to have a ground truth.
Because of these two shortcomings, we have recorded continuous real-world data sets, together with a third persion point of view video stream.
All the recorded data is made publicly available as the \gls{almende-data} data set \cite{vlasveld2014acras}.
Using our results as a benchmark, it is possible for others to use our data set and improve on it.

\section{Limitations}\label{sec:limitations}
Our proposed method \gls{ocs-hats} was constructed under the assumption that an explicit temporal segmentation of inertial sensor data can aid the process of activity classification.
We have expressed the problem of temporal segmentation as finding change points in the time series data.
Since this is the first application of \gls{oc-svm} based algorithms to real-world accelerometer data, we have limited ourselves to creating an initial method, without optimization.
The (results of the) proposed method is subject to parameters, which we have set manually based on empirical results and experience.

We have applied our method to real-world data, recorded in 7 runs.
For a better and more robust analysis of the algorithm it can be applied to other real-world data sets.
Although we have recorded and publicized our data, it would be nice if others also record continuous data sets with ground truth labeling, for fair comparison.
However, due to the application domain and ambiguity, the results remain subjective.

We have impplemented \gls{ocs-hats} in \textsc{Matlab}.
Because of that, we can not make reliable claims on the speed and memory usage of the algorithm on real-world devices such as smartphones.
With the online nature and applications in mind and different targeted hardware platforms, this is something that could be of importance.

\section{Future research}\label{sec:further_research}
The motivation for this research was, amongst others, the assumption that an explicit temporal segmentation of inertial signals can be used to obtain a better classification of the performed activities.
There are two reasons for that.
First, since the segmentation adds information about sequential data points (whether they belong to the same class), the a priori knowledge about the distribution changes.
Second, the window of data used for model construction can be much larger (than the often used length of 1-3 seconds).
This enables a better mathematical model to be constructed.
In future research the performance of classification methods that use temporal segmentation can be compared with those that do not.
This would confirm or reject the assumption that temporal segmentation enables better classification.

As discussed in the previous section, the parameters for our proposed method (such as the \gls{svdd} settings and \gls{rbf} width $\sigma$), are set manually and globally.
For further research we advise to create an automated optimization method for the parameters.
Furthermore, to overcome the \emph{masking} effect discussed in \Cref{sec:real_world_results}, a method which determines the parameters locally should obtain better results.

Since temporal segmentation of inertial signal is not yet a very active field of study, more studies can be performed on this subject.
In our approach we have used (all) the raw data directly, to create a method independent of the domain.
In the field of activity recognition many methods make use of pre-processing steps, from which one step often is feature extraction from the raw data.
That, and other pre-processing steps, could be incorporated in explicit temporal segmentation of inertial signals.
Future research could focus on using those extracted features to aid the detection of change points.

Similar, feature selection if often used to use only the best dimensions of a signal.
We have encountered situations in which the exclusion of some sensors, \eg the accelerometer, yields better results.
This occurs mainly in situations where a rotation is part of the change (\eg walking around a corner).
Thus depending on the desired granularity of the segmentation, it can be beneficial to discard some of the time series data.
Applying feature selection to the sensors to restrict the data used for model construction could be a subject of further research.

Finally, the application of \gls{oc-svm} based methods to new types of data is an interesting subject for further research in itself.
It creates a powerful non-linear classification solution and combined with optimization techniques can reduce problems of relatively high dimensionality to a single (or a few) dimensions.

\section{Final words}\label{sec:final_remarks}
In this research we have constructed a method to segment inertial sensor time series data, from performed human activities.
To this end we have applied \gls{occ} and \gls{oc-svm} based method to our own recorded data sets.
We followed the approach by Camci~\cite{camci2010change} and used the theory and implementation of \gls{svdd} by Tax~\cite{tax2001one,Ddtools2013}.
Although the performance of the method heavily depends on user set parameters, the results are promising.
We believe that the temporal segmentation of time series data can aid the process of activity classification, which is a logical follow-up on this research.

We have created a continuous recorded data set \gls{almende-data}~\cite{vlasveld2014acras} from inertial sensors found in typical modern smartphones.
The transition periods between the activities are part of the time series data.
This makes the data set useful for research on segmentation.
Besides the manually annotated change points can the ground truth of the performed activity be determined by using the provided video recordings of the activities.

With our method for temporal segmentation of human activities, \gls{ocs-hats}~\cite{vlasveld2014hats}, we believe to have contributed to the field of machine learning in the context of \acrlong{har}.
