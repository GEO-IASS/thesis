% Chapter 1

\chapter{Introduction} % Main chapter title

\label{Chapter1} % For referencing the chapter elsewhere, use \ref{Chapter1} 

\lhead{Chapter 1. \emph{Introduction}} % This is for the header on each page - 
%perhaps a shortened title

%----------------------------------------------------------------------------------------

\section{Problem statement}
*** Give context problems, means to reach (e.g. monitoring of patients, 
getting conclusions over health and activity).
%----------------------------------------------------------------------------------------

\section{Literature review}
Look into earlier application mentioned in the literature to this kind of 
problems. Look for similarities in the problem and address where the 
techniques used fail or are not applicable.

*** Most of the activity recognition techniques can be classified in two 
different types. Some rely on a state-space model, in which the activities to 
be recognized are represented in a statistical manner. Bayesian networks, 
Finite State Machines and Hidden Markov Models can be considered among those. 
Other techniques process the directly as a pattern recognition task, such as 
Support Vector Machines, Neural Networks, Dynamic Time Warping, and Bayes and 
K-means clustering.

%----------------------------------------------------------------------------------------

\section{System architecture}

*** [GRAPHICS diagram: sensors --> signal pre-processing --> feature 
extraction --> analysis [* segmentation, * labeling] --> conclusions]


%----------------------------------------------------------------------------------------


\section{Learning \LaTeX{}}

