\documentclass[a4paper,10pt]{extarticle}
\usepackage[english]{babel}
\usepackage{fullpage}
\usepackage{hyperref}
\author{R.Q. Vlasveld}
\title{Intended Table of Contents Thesis}


\begin{document}
\maketitle
\tableofcontents

\section{Introduction}
This section introduces the subject, outlines the global goal of the thesis and explains the problem.
The main research-questions are introduced.
In the subsections the main components (global, not as an implementation) are introduced.

  \subsection{Thesis goal}
  The goal of the thesis will be discussed, e.g. "precise unsupervised on-line human activity recognition".
  Mention application to show why it is useful.

  \subsection{Results of earlier research}
  First, results of earlier project are outlined. Look at the data being used, the precision and error rates (shortly, no details) and look at the different concepts being discussed.
  This will give reason for the following subsections.

\section{Concepts}

  \subsection{Temporal pattern recognition}
  This will explain about what patterns are and what kind of patterns need to be recognized (long-term, short-term repetitive, etc)

	\subsection{Signal pre-processing}
  Describe how data is collected and processed. Include feature extraction, like PCA, FFT.
  Mention only shortly with meaning, no further explanation.

    \subsubsection{Signal fusion}

    \subsubsection{Signal smoothing}

    \subsubsection{Feature extraction}

	\subsection{Temporal Segmentation}
  What is segmentation, the purpose and what kind of methods are there.
  E.g. model-based, statistics-based. Diagram with types. Supervised / unsupervised.
  HMMs, Statistical, Bayesian. K-means, SOM

  \subsection{Temporal clustering}
  When is segmenting clustering? Explain difference, mention techniques (diagrams), Dynamic Time Warping, k-means (and variations), SOM, SVM, Naive Bayes, Neural Networks.
  This is the unsupervised part.




\section{Problem statement}
This section proposes a setup to improve (or: answer) the research-questions.
It will discuss techniques used in other projects and how implemented here.
In the subsections each used technique is further explained. (Only explained when really used)

  \subsection{Proposed method}
  Give an overview of the "pipeline" of the implementation.
  Give a diagram how data is retrieved, (pre-)processed, analyzed and conclusions over data are drawn.

  \subsection{PCA}
  Something about the workings of PCA and why it is relevant and used.
  Discuss how it is used in other projects and how it will be used here.

  \subsection{Hidden Markov Models}
  Something about the workings of HMMs and why it is relevant and used.

  \subsection{Dynamic Time Warping}
  if used

  \subsection{k-Means clustering}
  if used

  \subsection{Self-organizing maps}
  if used


\section{Implementation}
This section will discuss implementation details.
Hardware platform (if any), software used. Complexity, precision of metrics etc.

\section{Experiments, results}
This section will outline the precise experiments and gives the results of these.

\section{Discussion}
This section will discuss the results of the previous section.
It will relate the results to previous work.

\section{Conclusion}
Conclusion (answer research question), summary of results, feature work.


\end{document}